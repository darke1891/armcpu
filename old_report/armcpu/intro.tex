% $File: intro.tex
% $Date: Fri Jan 31 18:50:38 2014 +0800
% $Author: jiakai <jia.kai66@gmail.com>

\section{导论}
实验目标如下:
\begin{enumerate}
	\item 使用老师提供的开发板,
		在FPGA上编程实现一个基于标准32位MIPS指令集的子集的流水CPU,
		支持异常、中断、TLB等。
	\item 在该CPU上运行ucore操作系统,进入用户态及shell环境,
		正常执行shell命令。
	\item 修改ucore,实现简单的远程文件执行功能,
		即通过串口从PC上获取ELF文件,并在本地执行。
	\item 修改编译原理课程中的decaf编译器,并结合GNU Binutils,
		编译出可在ucore上执行的ELF文件。
	\item 可选实现对VGA、ps/2 keyboard等其它外设的支持。
\end{enumerate}

实验硬件环境为老师提供的一块开发板,
主要核心部件包含Xilinx Spartan6 xc6slx100 FPGA,
8MB 32-bit字长的RAM,8MB 16-bit字长的flash,
另有一块与FPGA相连的CPLD用于一些外设I/O操作。

项目使用git管理,并托管在\url{https://git.net9.org/armcpu-devteam/armcpu}上。
目录结构如\figref{dirtree}所示。注意,项目中很多地方使用了软链接,
并且有大量的脚本,因此只能在linux等类unix系统下使用。

\begin{figure}[!ht]
\dirtree {%
.1 /.
.2 archive\DTcomment{经测试可用的FPGA, CPLD及ucore二进制映象}.
.2 armcpu.
.3 ipcore\_dir\DTcomment{xilinx ipcore相关,包括乘法器和VGA缓存}.
.3 prog\DTcomment{写入ROM的引导程序的汇编代码}.
.3 simulate\DTcomment{用于逻辑仿真的代码及期望输出}.
.3 src\DTcomment{CPU核心部分源代码}.
.4 gencode\DTcomment{用于代码生成的脚本}.
.3 synthesis\DTcomment{用于逻辑综合的代码及ucf文件}.
.2 decaf\DTcomment{修改过的decaf编译器}.
.2 lib\DTcomment{一些通用的ucf及verilog文件,提供常见设备的驱动}.
.2 misc\_fpga\_proj\DTcomment{开发初期进行实验及硬件测试的xilinx项目}.
.3 cpld\DTcomment{需要烧入CPLD的程序,用于转发串口及处理ps/2键盘扫描码}.
.2 report\DTcomment{实验报告的源代码}.
.2 screenshots.
.2 slides\DTcomment{展示时所用的幻灯片的相关资源}.
.2 ucore\DTcomment{修改过的ucore源代码}.
.3 ours\DTcomment{运行于ucore上的独立程序}.
.2 utils\DTcomment{各种工具脚本}.
}
\caption{\label{fig:dirtree}项目目录结构}
\end{figure}

% vim: filetype=tex foldmethod=marker foldmarker=f{{{,f}}}

