% $File: cpu.tex
% $Date: Fri Mar 14 21:33:03 2014 +0800
% $Author: jiakai <jia.kai66@gmail.com>

\section{CPU核心}
\subsection{整体设计}
CPU的整体设计如\figref{cpuarch}所示。其核心部分有instruction fetch(IF),
instruction decoding(ID), execution(EX), memory access(MEM)四个stage,
并有独立的multiplication(MULT), forwarding(FWD), memory management unit(MMU)
三个额外的辅助单元。

CPU的总体设计参考了\cite{bryant2008computer}和\cite{patterson2008computer}
中的相关章节。其基本思路是在不同的stage间插入锁存器,
其中有后续stage所需的全部信息,stage内部是组合逻辑,这样每经过一个时钟周期,
所以指令便会向后``移动''一个stage,从而增加throughput实现流水CPU。

寄存器堆(register file)直接放在了ID中,
这样便于在指令译码时访问寄存器得到相应的寄存器值。
对CP0和乘法器的直接读写都放在了MEM中进行,
这样可以在每个周期中向访问内存一样对其进行访问,使得整体设计更为统一。
另外,将CP0放在MEM中也有利于对异常和中断的处理。

\begin{figure}[!ht]
	\addplot{../cpuarch.png}
	\caption{\label{fig:cpuarch}CPU总体构架}
\end{figure}


\subsection{冲突处理}
流水线主要有结构冲突、数据冲突、控制冲突三种冲突类型。

由于指令和数据共用内存,而且没有实现cache,
结构冲突是不可避免的,目前我们采取了最简单的处理方式,
即访存时就设置stall标志位,暂停除MEM外所有stage的活动。

对于数据冲突,通过精心设计的forwarding单元已可完全解决,
无论前后指令间的数据依赖如何(甚至包括lw+jr等组合),均不需要暂停处理器。
其具体实现需要所有stage的共同协作和全局的设计考虑,不便于文字描述,
详见代码;需要注意的是,要充分利用时钟的正负边沿。

对于控制冲突,通过MIPS中的延迟槽即可解决,也无需暂停处理器,
但编译器可能需要插入不必要的NOP。

\subsection{异常处理}
MIPS要求精确异常处理(precise exception handling),
即要求流水CPU处理异常时对外所表现出的行为与单周期、多周期CPU一样。
为此,在相邻stage间加入\verb|exc_code|, \verb|exc_epc|,
\verb|exc_badvaddr|这三个锁存器,同时把CP0放在最后一个stage(即MEM)中。
在某个stage中检查到异常时,只将其到下一个stage的异常相关锁存器赋值,
而最后在MEM中才真正执行异常操作,即设置一些标志变量使得下一次IF
从异常处理或中断返回的位置取代码。这样一来,
异常发生前的所有指令都会被执行,而异常之后的指令都不会被执行,
从而实现了精确异常处理。

对于中断,MEM中的CP0判断中断条件是否成立,如果成立的话就设置相应标志变量,
使得下个周期中IF产生一个虚拟的中断异常,
待该异常传递到MEM后才进行实际的中断处理。

\subsection{内存管理}
虚拟内存方面,MMU单元负责进行指令和数据访存的切换,
TLB的维护及查找,以及产生内存相关异常。
MMU接收指令地址和数据地址两个输入,如果没有数据操作,则表现得像组合逻辑,
在半周期内向IF返回指令地址对应的内容或者访问其造成的异常;否则进行数据读写,
并在操作过程中设置输出的busy标志位,
待busy重新被置零后MEM就可以处理异常及数据。
另外也可以向MMU发送一个TLB写入指令,使得其写入一个TLB条目。

我们将CPU的边界定义在MMU。也就是说,对外而言,
CPU在每个周期内只会做以下几件事之一: 
\begin{itemize}
	\item CPU给出物理内存地址,要求在半个周期内得到相应数据。
	\item CPU给出物理内存地址及数据,启动物理内存写操作。
	\item CPU等待物理内存写操作的完成。
\end{itemize}

对于所有外设,均通过物理内存映射进行I/O。具体映射方案如\figref{iomap}所示。
\begin{figure}[!ht]
	\begin{center}\begin{tabular}{cll}
		\hline
		物理地址 & 对应设备 & 附加说明 \\ \hline
		\verb|[0x00000000, 0x007FFFFF]| & RAM & 共8MB \\ 
		\verb|[0x10000000, 0x10000FFF]| & 引导ROM & 共4KB \\
		\verb|[0x1E000000, 0x1EFFFFFF]| & flash & 
			地址空间共16MB,但对于每个32位的字,只有低16位可用 \\
		\verb|[0x1A000000, 0x1A096000]| & VGA & 详见\secref{vga} \\
		\verb|0x1FD003F8|	& 串口数据	& 可读写,读入则清除中断 \\
		\verb|0x1FD003FC|	& 串口状态	& 第0位为是否可写,第1位为是否可读 \\
		\verb|0x1FD00400|	& 数码管	&
			在2位数码管上显示所写入数据的最低有效字节 \\
		\verb|0x0F000000|	& 键盘码	& 读入则清除中断,详见\secref{ps2} \\
		\hline
	\end{tabular}\end{center}
	\caption{\label{fig:iomap}物理内存映射}
\end{figure}
如果访问到了未映射的物理地址,对于读操作始终返回0,对所有写操作忽略。

\subsection{实现及开发技巧}
\begin{itemize}
	\item {\bf 开发语言}  \\
		使用verilog而非VHDL作为开发语言,相比而言前者语法更为紧凑简洁,
		更具表达力,最后CPU的核心代码不过一千多行,而且也更贴近C的习惯,
		极大地减少了开发和维护的工作量。
	\item {\bf 元编程} \\
		由于xilinx相关工具不支持SystemVerilog,
		而后者具有结构体等很有用的语言特性,
		因此部分verilog代码使用python脚本生成以模拟类似C的struct结构;
		这部分代码主要用于跨stage的锁存器。
		另外,使用xilinx ipcore生成ROM的流程较为复杂也不易维护,
		而且实际ROM中的数据很少,因此ROM也采取python脚本生成,
		直接作为一个case语句放在verilog源代码里。
	\item {\bf 逻辑仿真} \\
		在CPU源代码中,使用verilog的\verb|$display|,\verb|$monitor|等system
		task输出了大量调试信息,再加上对串口、RAM、flash等设备的HDL模拟,
		并使用bash脚本作为入口点,最终实现了自动化的仿真工具链,
		只需给出汇编代码,就可以自动编译并将二进制载入模拟的RAM执行,
		同时dump出所有信号的波形文件,
		极大地方便了开发和调试。\figref{simulate}展示了仿真及波形显示的界面,
		其中波形显示使用了gtkwave。
		\begin{figure}[!ht]
			\addplot{../simulate.png}
			\caption{\label{fig:simulate}逻辑仿真的相关界面}
		\end{figure}
\end{itemize}

% vim: filetype=tex foldmethod=marker foldmarker=f{{{,f}}}

